%----------------------------------------------------------------------------------------
%	INTRODUZIONE
%----------------------------------------------------------------------------------------

\section{Introduzione}

La sostenibilità in Europa è un tema molto sentito anche nel settore alberghiero, da anni infatti sono presenti sistemi e centraline che permettono di ridurre gli sprechi energetici andando ad esempio a spegnere gli impianti elettrici all'uscita dell'ospite dalla sua camera.

Negli USA invece la maggior parte degli hotel non dispone di questi semplici meccanismi e l'utenza è molto meno attenta al problema eco sostenibilità. Un banale esempio di spreco energetico che si presenta con un'elevata frequenza, è quello in cui l'ospite, specialmente d'estate, esce dalla camera lasciando il condizionatore acceso a temperature molto basse e contemporaneamente le tende aperte.

Dal 2020 la \href{https://sustainablehospitalityalliance.org/}{Sustainable Hospitality Alliance}, che comprende il 30\% dell'industria alberghiera globale, si impegna a proporre linee guida e \textit{best practice} per il design e la gestione di hotel sostenibili. Tuttavia il processo per la sostenibilità è molto lungo ed impegnativo e l'alleanza è ancora in una fase preliminare di raccolta dati.

Inoltre il World Tourism Organization ha divulgato un documento che riassume alcuni atteggiamenti utili a ridurre i consumi prodotti dal turismo. Secondo un rapporto della Commissione Europea (‘Consumption and Environment’), il settore economico del turismo si colloca al quarto posto pre fabbisogno energetico. Infine uno studio della Confesercenti rilasciato ad agosto 2022, prevede un aumento delle bollette per gli hotel di oltre 1.5 miliardi di euro nei prossimi 12 mesi.

Il nostro obiettivo è quello di proporre all'alleanza sopracitata Ecotrip, un nuovo sistema che faccia leva direttamente sugli ospiti promuovendone un comportamento virtuoso in termini di consumi energetici/idrici.
L'idea è sviluppata in collaborazione con un nostro partner che progetta e realizza sia la struttura che gli interni di hotel e catene di hotel negli USA.

Il sistema è composto da una centralina, da installare nella camera d'albergo, che attraverso diversi sensori rileva i consumi di energia ed acqua e li carica su infrastruttura cloud.
In particolare la centralina misura i consumi di energia elettrica relativi al circuito luci e al circuito prese, il flusso di acqua fredda e calda con relative temperature,
la temperatura e umidità d'ambiente ed infine è in grado di determinare se le tende sono aperte o chiuse con sensori di luminosità ambientale.

Gli ospiti attraverso un'apposita applicazione su smartphone ottengono un "punteggio sostenibilità" calcolato in base al comportamento tenuto durante il proprio periodo di pernottamento, il punteggio 
può essere convertito in uno sconto al checkout. Per accedere ai dati del proprio pernottamento, all'ospite è richiesto semplicemente di avvicinare il suo smartphone ad un tag NFC installato all'interno della sua stanza.
Il tag è collegato e gestito dalla centralina che lo riprogramma per ogni pernottamento inserendovi il relativo token di accesso, garantendo così la privacy degli ospiti che saranno in grado di accedere solo ai propri dati.

Infine, il gestore dell'albergo dispone di un pannello di controllo per monitorare i dati delle varie stanze e gestire i pernottamenti.

Il sistema Ecotrip è pensato per essere installato in aggiunta alle centraline ed impianti già esistenti nelle stanze degli Hotel. 
Con Ecotrip l'ospite viene direttamente sensibilizzato alla eco-sostenibilità grazie a concetti di "gamification".

%Il contributo tecnologico apportato dal progetto si riassume in una centralina (\textit{control unit}) installa in ogni stanza, una piattaforma in \textit{cloud} per la raccolta dei dati e un' applicazione lato smartphone che permette di mostrare all'ospite un "punteggio sostenibilità" che ne evidenza l'attitudine al corretto consumo delle risorse o viceversa allo spreco. La centralina consiste in un computer/micro-controllore che esegue un servizio sviluppato ad-hoc, il quale interroga periodicamente sensori dislocati nella stanza. Questi rilevano fattori ambientali come illuminazione e temperatura della stanza ma anche consumi elettrici/idrici.

Il contributo tecnologico apportato dal progetto comprende:
\begin{itemize}
    \item l'ingegnerizzazione della centralina che si occupa di rilevare periodicamente fattori ambientali e consumi elettrici/idrici tramite sensori opportunamente connessi e programmati; questa si occupa anche di inviare i dati campionati sulla piattaforma cloud e di gestire il tag NFC. 
    \item Impiego della piattaforma cloud Amazon AWS per realizzare un'infrastuttura a microservizi che permetta la gestione delle centraline (AWS IoT Core), memorizzare i dati (AWS DynamoDB), gestire gli account/autenticazioni (AWS Cognito), calcolare i punteggi sostenibilità e generare le autorizzazioni per i visitatori con funzioni \textit{lambda}, ed infine fornire il pannello di controllo e l'app (AWS EC2 / S3).
    \item Sviluppo del pannello di controllo, applicativo composto da frontend (React) e backend (Nodejs) utilizzata sia dagli amministratori di Ecotrip, sia dagli albergatori. 
    \item Sviluppo web app (React) che consente all'ospite di visualizzare il proprio "punteggio sostenibilità" associato ad un determinato pernottamento.
\end{itemize}

\newpage
