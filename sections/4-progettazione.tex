%----------------------------------------------------------------------------------------
%	PROGETTAZIONE
%----------------------------------------------------------------------------------------

\section{Progettazione}
Una volta chiarificati tutti i requisiti ed i desiderata del cliente, si è passati alla fase di progettazione. Inizialmente si è preferito astrarre dalle tematiche di "basso livello" dando priorità alla definizione di tutti i servizi necessari per il corretto funzionamento del sistema. Nello specifico si parla di microservizi, infatti ogni componente software da sviluppare modella un insieme specifico di concetti, precedentemente identificati nei 9 sottodomini. Un aspetto di cruciale importanza è definire a monte le modalità di interazione tra i vari servizi. Per ognuna di queste è necessario specificare il "fornitore" (\textit{upstream}) ed il "consumatore" (\textit{downstream}), così da stabilire una dipendenza tra le coppie di servizi che vincoli una delle parti ad adattarsi alle regole operative dell'altra.

\begin{figure}[H]
    \includegraphics[width=\textwidth]{context-map.png}
    \centering
    \caption[contextmap]{Schema dei microservizi che compongono Ecotrip}
    \label{fig:contextmap}
\end{figure}

Le componenti del sistema Ecotrip (Figura \ref{fig:contextmap}) possono essere schematizzate in:
\begin{itemize}
    \item \textbf{Administration}: servizio che modella  i concetti e implementa le funzionalità di gestione degli hotel (\textit{Hotel Management}) e dei pernottamenti (\textit{Stay Management}).  Entrambe queste API necessitano di verificare l'autenticità delle richieste, per questo Administration dipende sia da AWS Cognito e che da Guest Authorization Service, il quale genera i \textit{token} usati da Guest App per eseguire le richieste. Queste due dipendenze in realtà non rappresentano connessioni con i servizi remoti in quanto le verifiche possono essere eseguite localmente ad Administration, tuttavia il processo di validazione è vincolato alle tecnologie usate dai rispettivi servizi. La comunicazione verso i servizi di AWS Cognito e Guest Authorization sarà regolamentata da opportuni \textit{adapter} (ACL) che consentono di convertire, sia in entrata che in uscita, i concetti di dominio esterni in quelli interni al servizio stesso.  
    Poiché è richiesto che il pannello di controllo sia fruibile via web, si applica il \textit{backend-for-frontend pattern}: Administration si occuperà di fornire una RESTful API ad uso del \textit{frontend} che sarà descritto in seguito.
    Si è quindi evitata una logica a microservizi pura in quanto:
    \begin{itemize}
        \item il ciclo di vita può essere unito in modo da avere un unico deployment su unica infrastruttura riducendo i costi di fornitura del servizio;
        \item Hotel Management avrà un numero di richieste sempre molto basso e lo \textit{scaling} del servizio può essere dimensionato pensando solamente a Stay Management.
    \end{itemize}
    L'alternativa a questa strategia è di separare i due servizi e magari implementarli con paradigma totalmente \textit{serverless} (vedi capitolo 7 - deployment), che renda i costi del servizio proporzionali all'utilizzo.
    \item \textbf{AWS Cognito}: rappresenta il sottodominio relativo all'autenticazione, cioè tutte le funzionalità che consentono di gestire gli \textit{account} degli amministratori di sistema e degli \textit{hotelier}, che può essere demandato in \textit{out-sourcing}. La soluzione da noi scelta è appunto Amazon AWS Cognito.
    \item \textbf{Control Panel}: realizza la logica di business \textit{frontend} dei sottodomini Administration e Authentication, nella pratica si andrà a realizzare una \textit{web app} sfruttando il framework React.js. Questa delegherà l'autenticazione e la gestione degli account creati ad AWS Cognito. Inoltre, il pannello di controllo risultante dovrà permettere di visualizzare sia lo stato dalle centraline che i dati prodotti da queste. Ciò comporta la creazione di un collegamento tra il Control Panel ed i servizi AWS IoT Core e Data Elaboration Service.
    \item \textbf{AWS IoT Core}: include i sottodomini Control Unit Management, Control Unit Maintenance e lo \textit{storage cloud} dei dati generati da Room Monitoring. Come per l'autenticazione, la sua implementazione può essere delegata a un fornitore esterno, nel caso specifico si è scelto l'omonimo servizio di amazon AWS IoT Core. Quest'ultimo dispone di un pannello di controllo con cui poter agevolmente eseguire tutti i \textit{task} richiesti, per esempio permette di abbinare le centraline alle stanze attraverso \textit{tagging}. Infine, tramite un'apposita Rest API, per ogni centralina si è in grado di ottenere i dati caricati, verificarne lo stato ed inviare dei comandi da remoto: funzionalità che può essere sfruttata per inviare il token di un ospite.
    \item \textbf{Guest Authorization}: realizza funzionalità che possono essere circoscritte in due sistemi distinti. Il primo è Guest Authorization Service, il quale si occupa della generazione del \textit{token}, a seguito della creazione di un nuovo soggiorno (eventi di \textit{check in} e \textit{check out}) da parte del servizio di Stay Management, e del suo invio verso la centralina tramite AWS IoT Core. Il secondo sistema verrà incluso come modulo all'interno del software della centralina.
    \item \textbf{Data Elaboration}: servizio che si attiva periodicamente per elaborare i dati raccolti per ogni pernottamento e storicizzare i calcoli dei consumi e punteggio sostenibilità. Fornisce anche un API per fornire i dati a Guest App e Control Panel.
    \item \textbf{Control Unit}: rappresenta il software installato sulla centralina, il quale si compone di due moduli, il primo (Room Monitoring) si occupa del monitoraggio dei consumi, mentre il secondo (Guest Authorization) è dedicato alla gestione del \textit{transponder NFC}. La Control Unit oltre a ricevere gli aggiornamenti di stato da AWS IoT Core, vi inoltra i dati raccolti dai sensori. Come fatto per Administration, anche in questo caso si è evitato di creare due software indipendenti al fine di avere lo stesso ciclo di vita e un unico \textit{deployment} che possa semplificare le attività di configurazione e manutenzione della centralina.
    \item \textbf{Guest App}: realizza un applicativo lato \textit{frontend} che comunica con la Control Unit per ottenere il \textit{token} da presentare al servizio di Data Elaboration per accedere ai dati del pernottamento.
\end{itemize}

Di seguito viene proposto un approfondimento per quanto riguarda le componenti più complesse del sistema.

Si noti che ciascuno dei servizi di cui sopra gestisce autonomamente i propri dati, in modo separato l'uno dall'altro, per questo non abbiamo realizzato un modello dei dati composito.

Si rimanda inoltre al capitolo 7 relativo al deployment che è parte integrante della presente progettazione,
in quanto i servizi e gli applicativi web sono stati pensati considerando fin da subito a come questi possano effettivamente 
interagire tra loro ed essere resi pubblicamente disponibili.

\subsection{Control Unit}
Nel caso specifico della centralina la fase di progettazione coinvolge sia la componente \textit{hardware} che quella \textit{software}. L'elemento \textit{core} del prototipo è una scheda \textit{raspberry pi 4B} alla quale sono connessi i sensori precedentemente citati nella sezione dei requisiti. Data la finalità prototipale e le tempistiche ristrette si predilige l'acquisto di un set di sensori che siano disponibili nell'immediato, preferibilmente digitali e con un supporto software adeguato. Ovviamente, di questi si verificherà anche l'accuratezza nonostante non sia ritenuto un aspetto fondamentale durante questa fase di progetto. 

\subsubsection{Hardware}
Dopo un'attenta fase di ricerca e un confronto tra le diverse soluzioni, si opta per l'acquisto di un determinato set di sensori, presentato e approfondito nella sezione corrente. Di ciascuno verrà fornita una breve descrizione che include le caratteristiche tecniche principali, mentre i dettagli relativi alla componente software saranno trattati nella sezione successiva.\newline\newline
%
\textbf{ICQUANZX PN532 NFC}\footnote{Link al venditore: \href{https://www.amazon.it/ICQUANZX-Communication-Arduino-Raspberry-Android/dp/B07VT431QZ/}{https://www.amazon.it/ICQUANZX}}: modulo di trasmissione altamente integrato per la comunicazione NFC (Near Field Communication) a 13.56 MHz. Il sensore è dotato di un interruttore con il quale è possibile cambiare la modalità scegliendo tra I2C, SPI e UART. Inoltre, il modulo supporta la lettura/scrittura RFID e possiede 4 fori di montaggio da 3mm. 
    
\begin{table}[H]
    \centering
    \begin{tabular}{|l|l|}
    \hline
    \textbf{Caratteristica}     & \textbf{Descrizione}                      \\ \hline        Protocollo di comunicazione                   & I2C\footnote{Per dettagli: \url{https://www.i2c-bus.org/}}, SPI and HSU (High Speed UART)                                                                                                                     \\ \hline
    Supporto RFID                                 & \begin{tabular}[c]{@{}l@{}}modalità in lettura o scrittura:\\   - Mifare 1k, 4k, Ultralight, and DesFire cards\\   - ISO/IEC 14443-4 card\end{tabular} \\ \hline
    Distanza di comunicazione                     & 5cm$\sim$7cm (PCB Antenna)                                                                                                                             \\ \hline
    Virtualizzazione                              & può funzionare come una carta virtuale                                                                                                                 \\ \hline
    Dimensioni                                    & 43mm x 41mm x 4mm                                                                                                                                          \\ \hline
    \end{tabular}
    \caption{\label{pn532-features}Caratteristiche principali del modulo PN-532.}

\end{table}

La modalità di funzionamento determina anche la configurazione a livello di circuito (alimentazione).
\begin{table}[H]
    \centering
    \begin{tabular}{|c|c|}
    \hline
    \textbf{Interfaccia} & \textbf{Valore}                            \\ \hline
    VCC                & 3.3V$\sim$5V                              \\ \hline
    I2C/UART           & 3.3V$\sim$24V TTL                         \\ \hline
    SPI                & 3.3V TTL con 100 ohm di resistenza \\ \hline
    \end{tabular}
    \caption{\label{pn532-modalities}Modalità di funzionamento del modulo PN-532.}
\end{table}

\begin{figure}[H]
    \begin{center}
        \includegraphics[width=0.35\textwidth]{images/sensors/pn532.png}
    \end{center}
    \caption{\label{pn532}Modulo PN532}
\end{figure}

Nel caso si voglia comunicare con il sensore tramite l'interfaccia I2C, è necessario impostare gli interruttori rispettivamente a \texttt{ON} e \texttt{OFF} (Tabella \ref*{tb-PN532-interfaces}). Inoltre la connessione alla scheda richiede l'utilizzo di 4 pin (Tabella \ref*{tb-PN532-conn-schema}). Una volta connesso il modulo sarà identificato dalla scheda con l'indirizzo \texttt{0x24}.
\begin{table}[H]
    \centering
    \begin{tabular}{|l|l|l|}
    \hline
    \textbf{Interfaccia} & \textbf{Canale 1} & \textbf{Canale 2} \\ \hline
    HSU                  & OFF               & OFF               \\ \hline
    I2C                  & ON                & OFF               \\ \hline
    SPI                  & OFF               & ON                \\ \hline
    \end{tabular}
    \caption{\label{tb-PN532-interfaces}Schema di collegamento modulo PN532.}
\end{table}

\begin{table}[H]
    \centering
    \begin{tabular}{|l|l|l|}
    \hline
    \multicolumn{1}{|c|}{\textbf{Module PCB}} & \multicolumn{1}{c|}{\textbf{Desc}} & \multicolumn{1}{c|}{\textbf{GPIO Header Pins}} \\ \hline
    GND                                       & Ground                             & P1-6                                          \\ \hline
    SDA                                       & I2C SDA                            & P1-03                                          \\ \hline
    SCL                                       & I2C SCL                            & P1-05                                          \\ \hline
    VCC                                       & 3.3V                               & P1-01                                          \\ \hline
    \end{tabular}
    \caption{\label{tb-PN532-conn-schema}Schema di collegamento modulo PN532.}
\end{table}

\begin{figure}[H]
    \begin{center}
      \includegraphics[width=0.6\textwidth]{images/sensors/pn532-fritzing.png}
    \end{center}
    \caption{\label{pn532-diagram}Diagramma Fritzing del modulo PN532 (connessione I2C).}
\end{figure}
    
    
\textbf{GY-302 BH1750}\footnote{Link al venditore: \href{https://www.amazon.it/AZDelivery-Sensore-Intensità-Luminosità-Raspberry/dp/B07NLL4SCB}{https://www.amazon.it/AZDelivery-BH170}}: sensore digitale per il rilevamento della luminosità ambientale basato sull'interfaccia bus I2C; internamente monta un ADC. Le caratteristiche complete del sensore possono essere visionate consultando la pagina del \href{https://www.mouser.com/datasheet/2/348/bh1750fvi-e-186247.pdf}{\textit{datasheet}}. 

\begin{table}[H]
    \centering
    \begin{tabular}{|l|l|}
    \hline
    \textbf{Caratteristica}     & \textbf{Descrizione}                      \\ \hline    Alimentazione               & 3$\sim$5V                                 \\ \hline
    Protocollo di comunicazione & I2C                                       \\ \hline
    Output                      & segnale in uscita digitale (ADC build-in) \\ \hline
    Accuratezza                 & precisione elevata, vicino a 1 Lu         \\ \hline
    Data range                  & 0$\sim$65535                              \\ \hline
    Dimensioni                  & 13.9 x 18.5 mm                           \\ \hline
    \end{tabular}
    \caption{\label{bh1750-features}Caratteristiche principali del modulo BH1750.}
\end{table}
%
\begin{figure}[H]
    \begin{center}
      \includegraphics[width=0.35\textwidth]{images/sensors/bh1750.png}
    \end{center}
    \caption{Modulo BH1750}
\end{figure}
%
Il sensore può essere collegato al Raspberry Pi semplicemente sfruttando 4 collegamenti (o meglio 4 pin GPIO della scheda come descritto in Tabella \ref*{tb-BH1750-conn-schema}). Una volta realizzato il circuito (Figura \ref*{bh1750-diagram}), il comando \texttt{i2cdetect} restituirà in input l'indirizzo del sensore, che nel caso specifico equivale a \texttt{0x23}.
%
\begin{table}[H]
    \centering
    \begin{tabular}{|l|l|l|}
    \hline
    \multicolumn{1}{|c|}{\textbf{Module PCB}} & \multicolumn{1}{c|}{\textbf{Desc}} & \multicolumn{1}{c|}{\textbf{GPIO Header Pins}} \\ \hline
    GND                                       & Ground                             & P1-14                                          \\ \hline
    ADD                                       & Address select                     & P1-14                                          \\ \hline
    SDA                                       & I2C SDA                            & P1-03                                          \\ \hline
    SCL                                       & I2C SCL                            & P1-05                                          \\ \hline
    VCC                                       & 3.3V                               & P1-01                                          \\ \hline
    \end{tabular}
    \caption{\label{tb-BH1750-conn-schema}Schema di collegamento modulo BH1750.}
\end{table}

\begin{figure}[H]
    \begin{center}
      \includegraphics[width=0.6\textwidth]{images/sensors/BH1750-fritzing.png}
    \end{center}
    \caption{\label{bh1750-diagram}Diagramma Fritzing del modulo BH1750.}
\end{figure}

%
\textbf{ACS712 - 20A}\footnote{Link al venditore: \href{https://www.amazon.it/AZDelivery-corrente-sensore-Current-Raspberry/dp/B0736DYV3W?th=1}{https://www.amazon.it/AZDelivery-ACS712}}: sensore di corrente usato per misurare una corrente AC o DC in un range di ±20A con un errore di 1.5\% a T = 25 °C. Il sensore è composto da due parti, un collegamento per il chip del sensore, e l'altra parte con due connettori a morsettiera per la misurazione della corrente.
Il sensore utilizza l'effetto Hall per rilevare la corrente che lo attraversa. La corrente che fluisce attraverso il sensore genera un campo magnetico che viene rilevato dal sensore e convertito in una tensione analogica proporzionale. Questo modulo emette tensione analogica (0 ÷ 5 V) in base al flusso di corrente nel filo su cui viene eseguita la misurazione.
%
\begin{table}[H]
    \centering
    \begin{tabular}{|l|l|}
    \hline
    \textbf{Caratteristica} & \textbf{Descrizione}                                \\ \hline
    Alimentazione           & 5V                                                  \\ \hline
    Tensione in uscita      & metà di quella di alimentazione (2.5V se Vcc = 5V) \\ \hline
    Output                  & segnale in uscita digitale (ADC build-in)           \\ \hline
    Dimensioni              & 89.9 x 59.9 x 20.1 mm                               \\ \hline
    \end{tabular}
    \caption{\label{acs712-features}Caratteristiche principali del modulo ACS712.}
\end{table}

\begin{figure}[H]
    \begin{center}
      \includegraphics[width=0.35\textwidth]{images/sensors/acs712.png}
    \end{center}
    \caption{Modulo ACS712}
\end{figure}

\textbf{DHT22}\footnote{Link al venditore: \href{https://www.amazon.it/AZDelivery-temperatura-circuito-Raspberry-gratuito/dp/B078SVZB1X/ref=sr_1_6?keywords=dht22&qid=1673890614&sr=8-6&th=1}{https://www.amazon.it/AZDelivery-DHT22}}: sensore di umidità/temperatura relativa che emette un segnale digitale. Utilizza un sensore di umidità capacitivo e un termistore per misurare l'aria circostante. Nello specifico, la componente di umidità è costituita da un substrato di trattenimento che assorbe il vapore acqueo, andando ad aumentare la conduttività di due elettrodi posti agli estremi. Mentre la parte di rilevamento della temperatura del dispositivo è costituita da un sensore di temperatura NTC, cioè un termistore la cui resistenza diminuisce con l'aumentare della temperatura. Infine, c'è un piccolo PCB con un IC (circuito integrato) che esegue la conversione da analogico a digitale ed emette la coppia di valori (umidità e temperatura).

\begin{table}[H]
    \centering
    \begin{tabular}{|l|l|}
    \hline
    \textbf{Caratteristica} & \textbf{Descrizione}               \\ \hline
    Alimentazione           & 3$\sim$5V                          \\ \hline
    Tensione operativa max  & 2.5mA max                          \\ \hline
    Range umidità           & 0\% - 100\%  (accuratezza 2 - 5\%) \\ \hline
    Range di temperatura    & -40°C - 125°C (accuratezza ±0.5°C) \\ \hline
    Freq. di campionamento  & 0.5Hz (lettura ogni 2s)            \\ \hline
    Dimensioni              & 15 x 38 x 9 mm                     \\ \hline
    \end{tabular}
    \caption{\label{DHT22-features}Caratteristiche principali del modulo DHT22.}
\end{table}

\begin{figure}[H]
    \includegraphics[width=.35\textwidth]{images/sensors/dht22-a.png}\hfill
    \includegraphics[width=.35\textwidth]{images/sensors/dht22-b.png}\hfill
    \caption{Modulo DHT22.}
\end{figure}

Il dispositivo richiede solamente tre collegamenti con il Raspberry Pi: +5v, terra e un pin GPIO (Tabella \ref*{tb-dht22-conn-schema}).

\begin{table}[H]
    \centering
    \begin{tabular}{|l|l|l|}
    \hline
    \multicolumn{1}{|c|}{\textbf{Modulo PCB}} & \multicolumn{1}{c|}{\textbf{Descrizione}} & \multicolumn{1}{c|}{\textbf{GPIO Pin}} \\ \hline
    GND                                       & Ground                                    & P1-30                                  \\ \hline
    SCL                                       & GPIO 12                                   & P1-32                                  \\ \hline
    VCC                                       & 5V                                        & P1-02                                  \\ \hline
    \end{tabular}
    \caption{\label{tb-dht22-conn-schema}Schema di collegamento modulo DHT22}
\end{table}

\begin{figure}[H]
    \begin{center}
      \includegraphics[width=0.6\textwidth]{images/sensors/DHT22-fritzing.png}
    \end{center}
    \caption{\label{dht22-diagram}Diagramma Fritzing del modulo DHT22.}
\end{figure}

\textbf{ADS1115}\footnote{Link al venditore: \href{shorturl.at/btJVZ}{https://www.amazon.it/AZDelivery-ADS1115}}: convertitore da analogico a digitale, facilmente utilizzabile con una scheda Raspberry Pi utilizzando un bus di comunicazione I2C. Il convertitore ha una buona precisione poiché lavora a 16 bit e sfrutta 4 canali. Le caratteristiche complete del sensore possono essere visionabili alla pagina del \href{https://cdn-shop.adafruit.com/datasheets/ads1115.pdf}{datasheet}. 
Il convertitore utilizza il protocollo di comunicazione I2C per diverse ragioni:
\begin{itemize}
    \item I2C è un protocollo di comunicazione semplice e ampiamente utilizzato, il che lo rende facile da interfacciare con una vasta gamma di microcontrollori e computer singola scheda come Raspberry Pi;
    \item I2C è un protocollo \textit{two-wire}, cioè sono sufficienti solo due fili per il trasferimento dati e la sincronizzazione del \textit{clock}, rendendolo facile da implementare e riducendo il numero di connessioni necessarie;
    \item I2C supporta più dispositivi sullo stesso bus, consentendo a più ADCs di essere collegati allo stesso microcontrollore o computer singola scheda.
\end{itemize}
    
\begin{table}[H]
    \centering
    \begin{tabular}{|l|l|}
    \hline
    \textbf{Caratteristica}   & \textbf{Descrizione} \\ \hline
    Alimentazione             & 2.0$\sim$5.5V        \\ \hline
    Tipo di interfaccia       & I2C                  \\ \hline
    Bit rate ADC              & 16 bit               \\ \hline
    Canali                    & AN0, AN1, AN2, AN3   \\ \hline
    Amplificatore di guadagno & programmabile        \\ \hline
    Dimensioni                & 90 x 52 x 10 mm      \\ \hline
    \end{tabular}
    \caption{\label{ADS1115-features}Caratteristiche principali del modulo ADS1115.}
\end{table}

\begin{figure}[H]
    \centering
    \includegraphics[width=0.35\textwidth]{images/sensors/ads1115.png}\hfill
    \caption{Modulo ADS1115.}
\end{figure}
Il convertitore può essere collegato al Raspberry Pi sfruttando il protocollo I2C (Tabella \ref*{tb-ADS1115-conn-schema}).
\begin{table}[H]
    \centering
    \begin{tabular}{|l|l|l|}
    \hline
    \multicolumn{1}{|c|}{\textbf{Module PCB}} & \multicolumn{1}{c|}{\textbf{Desc}} & \multicolumn{1}{c|}{\textbf{GPIO Header Pins}} \\ \hline
    GND                                       & Ground                             & P1-20                                          \\ \hline
    SDA                                       & I2C SDA                            & P1-03                                          \\ \hline
    SCL                                       & I2C SCL                            & P1-05                                          \\ \hline
    VCC                                       & 3.3V                               & P1-01                                          \\ \hline
    \end{tabular}
    \caption{\label{tb-ADS1115-conn-schema}Schema di collegamento modulo ADS1115.}
\end{table}

\begin{figure}[H]
    \begin{center}
      \includegraphics[width=0.6\textwidth]{images/sensors/ads1115-fritzing.png}
    \end{center}
    \caption{\label{ads1115-diagram}Diagramma Fritzing del modulo ADS1115.}
\end{figure}

\textbf{SENSTREE G1/2 CF-B7}\footnote{Link al venditore: \href{https://www.amazon.it/Interruttore-Misuratore-Contatore-Flussometro-Temperatura/dp/B07QNMZ7ZK}{https://www.amazon.it/AZDelivery-CFB7}}: il sensore del flusso d'acqua è costituito da un corpo in rame, un rotore dell'acqua, un magnete e un sensore ad effetto hall. Quando l'acqua scorre attraverso il rotore, questo inizia a girare insieme al magnete. La rotazione del campo magnetico attiva il sensore Hall, il quale emette onde quadre cioè impulsi. Ad ogni giro si avrà un certo volume di acqua che scorre, così come un determinato numero di onde quadre emesse. Pertanto, possiamo calcolare il flusso d'acqua contando il numero di onde quadre.

La formula per il calcolo del flusso d'acqua equivale a:
\[l_{hour} = \frac{flow_{frequency} \cdot 60}{11}\]

Sapendo che, nel caso di CF-B7, il sensore Hall genera 660 impulsi per ogni litro d'acqua, quindi ogni impulso corrisponde a $\frac{1}{660}$ litri. Detto ciò è possibile ricavare il volume totale ($V_{total}(L)$) di liquido che fluisce attraverso il sensore ad un certo tempo $t$, sfruttando il numero di impulsi:
\[V_{total}(L) = N \cdot \frac{1}{660}(L) \]

Inoltre, il volume precedente può essere calcolato anche come $water flow rate(Q - unit L/s)$ moltiplicato per il tempo $t$:
\[V_total(L) = Q(L/s) \cdot t(s) \]

Quindi, si ottiene:
\begin{gather*}
    N \cdot 1/660 = Q(L/s) \cdot t(s) \\
    N/t = 660 \cdot Q(L/s)
\end{gather*}

Infine, siccome $\frac{N}{t}$ equivale alla frequenza $f$:
\begin{gather*}
    f = 660 \cdot Q(L/s) \\
    Q(L/s) = \frac{f}{660} \\
    Q(L/min) = \frac{f \cdot 60}{660} = \frac{f}{11} \\
    Q(L/hour) = \frac{f \cdot 60 \cdot 60}{660} = \frac{f \cdot 60}{11} 
\end{gather*}

\begin{table}[H]
    \centering
    \begin{tabular}{|l|l|}
    \hline
    \textbf{Caratteristica} & \textbf{Descrizione}                \\ \hline
    Alimentazione           & 5$\sim$15 V                         \\ \hline
    Pressione massima acqua & 1,75 MPa                            \\ \hline
    Portata                 & 1$\sim$25L/min                      \\ \hline
    Sensore di temperatura  & NTC 3950 R25 = 50 K.                \\ \hline
    Intervallo di errore    & (1$\sim$30L\textbackslash MIN) ±3\% \\ \hline
    Frequenza               & F=(11*Q) Q=L/MIN ±3\%                 \\ \hline
    Temperatura dell'acqua  & $\leq$ 120°C                              \\ \hline
    Dimensioni              & 65 x 30 x 28 mm                     \\ \hline
    \end{tabular}
    \caption{\label{tb-CF-B7-conn-schema}Caratteristiche principali del modulo CF-B7.}
\end{table}

\begin{figure}[H]
    \centering
    \includegraphics[width=0.35\textwidth]{images/sensors/cfb7.png}\hfill
    \caption{Modulo CF-B7.}
\end{figure}

Come mostrato in Tabella \ref*{tb-CF-B7-conn-schema} il flussometro è provvisto di un foro per l'inserimento di un termistore NTC (Negative Temperature Coefficient) 50K (\href{https://www.tme.eu/Document/34c96454d7432cb275d8954161fb18c2/NTCM-HP-50K-1.pdf}{datasheet}), la cui resistenza elettrica decresce all'aumentare della temperatura. Il sensore ha un ottimo rapporto qualità/prezzo e può essere trovato sul mercato in tante diverse configurazioni.
Per valutare un termistore, la temperatura standard utilizzata dalla maggior parte dei produttori è di 25 °C, che equivale a quella ambiente. Nel caso specifico, a 25° il sensore presenterà una resistenza elettrica di 50K ohm.

Per il calcolo della temperatura ci si affida alla formula semplificata di Steinhart-Hart, cioè alla \textit{B parameter equation}:
%
\[\frac{1}{T} = B \ln{\frac{R}{R_0}} + \frac{1}{T_0}\]
%
Dove $T$ è la temperatura in Kelvin, $R$ è la resistenza del termistore, $R0$ (50K ohm) è la resistenza del termistore a una temperatura nota $T0$ (25 °C) e $B$ è un costante specifica per il termistore (nel caso specifico 3950). 

Il sensore non può essere collegato direttamente al Raspberry perché il segnale deve prima essere convertito da analogico a digitale tramite l'ADS1115. Si è scelto di inserire una resistenza di 10K ohm così da aumentare la risoluzione della misura.
\begin{figure}[H]
    \begin{center}
      \includegraphics[width=0.6\textwidth]{images/sensors/ntc-fritzing.png}
    \end{center}
    \caption{\label{ntc50k-diagram}Diagramma Fritzing del modulo NTC 50K.}
\end{figure}

\subsubsection{Software}
Come già anticipato le funzionalità della control unit sono state raccolte nei sottodomini Guest Authorization e Room Monitoring. Il primo definisce la gestione del \textit{token} di autorizzazione mentre la seconda descrive la logica di calcolo ed invio dei consumi.
% Dato un discreto numero di concetti di dominio da modellare, ci si è avvalsi del \textit{domain model pattern}. Infatti la business logic viene espressa in termini di:

%     value objects: utilizzati per rappresentare il token di autorizzazione e le varie tipologie di misurazioni (temperatura, corrente, ecc.);
%     entities: sfruttate per definire il concetto di rilevazione (detection) e sensore;
%     domain services: impiegati per modellare la logica di persistenza del token corrente (TokenRepository).

% Il pattern adottato prevederebbe altri elementi costitutivi, come gli aggregators, ma il loro impiego non è stato necessario dato il livello di complessità del dominio.
Inoltre la struttura dell'applicativo è sostenuta da una architettura esagonale (Figura \ref*{clarc}), la quale garantisce caratteristiche quali:
\begin{itemize}
    \item \textit{modularità}: le regole operative possono essere collaudate indipendentemente dalla UI, dal database o qualsiasi altro elemento esterno;
    \item \textit{indipendenza dai framework}: la scelta dei framework ricade solamente sull'ultimo strato dell'architettura, così da utilizzare questi come semplici strumenti evitando di sottostare a specifici vincoli;
    \item \textit{indipendenza dal database}: la business logic non è legata nè a un singolo database nè ad una specifica tipologia.
\end{itemize}

Tutto questo è reso possibile dal rispetto della "regola della dipendenza", la quale sostiene che le dipendenze presenti nel codice sorgente devono puntare solo all'interno, verso le politiche di alto livello. Nella pratica, alcune classi degli strati più esterni vanno ad implementare interfacce definite in quelli più interni. Infatti, la comunicazione con i servizi esterni avviene per mezzo di \textit{adapter} (interfacce) descritti internamente ed opportunamente concretizzati nell'ultimo livello. Inoltre questa tipologia di architettura consente di delineare un confine netto tra i due sottodomini, separando questi in due moduli distinti.
%
\begin{figure}[H]
    \centering
    \includegraphics[width=\textwidth]{images/cl-architecture.png}\hfill
    \caption{\label{clarc}Rappresentazione grafica dell'architettura esagonale.}
\end{figure}
%Si può quindi dire che la progettazione della control unit è il risultato della combinazione della terminologia definita dall'ubiquitous language, con gli elementi del domain model pattern ed i concetti dell'architettura esagonale (Figura \ref{cu-uml}).%
Di particolare interesse è la definizione dello strato \textit{core} mediante casi d'uso, questi permettono di orchestrare i flussi di dati da e verso le entità, rimanendo aderenti agli schemi elaborati durante la fase di analisi. In Figura \ref{cu-uml} vengono schematizzati tutti i concetti principali di dominio, alcuni di questi sono semplici contenitori statici di valori (es. \texttt{Resistance}, \texttt{Current}, \texttt{Token}, ecc.) mentre altri definisco la logica dei casi d'uso e necessitano di un rapido approfondimento:
\begin{itemize}
    \item \texttt{EnvironmentUseCases}: racchiude la logica relativa al rilevamento dei fattori ambientali tramite i sensori di flusso dell'acqua, luminosità, temperatura e umidità;
    \item \texttt{ConsumptionUseCases}: come si evince dal nome, si occupa della raccolta dei consumi idrici ed elettrici;
    \item \texttt{AuthorizationUseCases}: set di funzionalità fondamentali per l'avvio della centralia, la ricezione/comunicazione del \textit{token} d'accesso e l'implementazione del protocollo di comunicazione tra la \textit{control unit} e uno smartphone nelle vicinanze.
\end{itemize}
\begin{figure}[H]
    \centering
    \includegraphics[width=\textwidth]{images/cu-uml.png}\hfill
    \caption{\label{cu-uml}Modellazione dei concetti di dominio tramite UML delle classi.}
\end{figure}
Nel caso delle rilevazioni, i dati prodotti vengono assegnati ad una specifica istanza di \texttt{Detection} che, oltre includere il valore rilevato, è associata alla data di creazione e ad un \textit{id} che la identifica univocamente. 
%
\begin{figure}[H]
    \centering
    \includegraphics[width=\textwidth]{images/room-monitoring-serivce.png}\hfill
    \caption{\label{rm-uml}Diagramma UML delle classi (\texttt{RoomMonitoringService}).}
\end{figure}
%
Infine la logica di monitoraggio della stanza (\textit{Room Monitoring}) è racchiusa all'interno della classe \texttt{RoomMonitoringService} (Figura \ref*{rm-uml}). Questa, tramite un proprio flusso di controllo, esegue periodicamente (ogni 5 secondi) ed in maniera concorrente tutti i rilevamenti necessari (consumi e ambiente), interrompendoli nel caso di superamento del \textit{timeout}. Una volta ottenuti tutti i risultati, questi vengono prima serializzati grazie un'istanza di \texttt{Serializer} e successivamente comunicati verso l'esterno tramite l'\texttt{OutputAdapter}. Data la natura concorrente del servizio, si è scelto di schematizzarne il comportamento attraverso un diagramma di stato (Figura \ref*{rm-uml-state}).

\begin{figure}[H]
    \centering
    \includegraphics[width=\textwidth]{images/room-monitoring-service-state-diagram.jpeg}\hfill
    \caption{\label{rm-uml-state}Diagramma UML di stato (\texttt{RoomMonitoringService}).}
\end{figure}

\subsection{Administration}

Per Administration si adotta un architettura semplice di tipo \textit{transaction script} sufficiente a gestire la complessità della sua logica di business.
In particolare questo si traduce in una Rest API dove ogni rotta avvia una singola procedura che si occupa della transazione.
Ciascuna transazione per essere eseguita richiede un token di autenticazione, generato dal servizio AWS Cognito. 
Inoltre vengono adottati meccanismi RBAC per autorizzare l'esecuzione delle transazioni in base alla tipologia di utente, amministratore o albergatore.

Riportiamo in tabella \ref*{administration-routes} le rotte previste ed in figura \ref*{er-admin} un semplice diagramma ER che modella i dati.

\begin{table}[H]
    \centering
    \begin{tabular}{|l|l|l|}
    \hline
    \textbf{Rotta}     & \textbf{Richiesta} & \textbf{Descrizione} \\ \hline
    /hotels/           & get & elenco degli hotels    \\ \hline
    /hotels/                     & post & aggiunta di un hotel   \\ \hline  
    /hotels/:id                     & get & dettagli di un hotel  \\ \hline 
    /hotels/:id                     & patch & modifica di un hotel \\ \hline
    /hotels/:id                     & delete & cancellazione di un hotel \\ \hline
    /hotels/:id/rooms           & get & elenco delle stanze di hotel    \\ \hline
    /hotels/:id/rooms           & put & aggiunta di una stanza ad un hotel    \\ \hline
    /hotels/:id/user           & get & per ottenere l'account utente dell'albergatore   \\ \hline
    /hotels/:id/user           & put & aggiunta l'account utente dell'albergatore   \\ \hline
    /rooms/:id                     & get & dettagli di una stanza  \\ \hline 
    /rooms/:id                     & patch & modifica di una stanza \\ \hline
    /rooms/:id                     & delete & cancellazione di una stanza \\ \hline
    /rooms/:id/stays               & get & elenco pernottamenti di una stanza \\ \hline
    /rooms/:id/stays               & put & check-in per una stanza \\ \hline
    /rooms/:id/currentStay         & get & dati pernottamento corrente di una stanza \\ \hline
    /rooms/:id/currentStay         & patch & check-out \\ \hline
    /guest/           & get & per ottenere i dati da visualizzare nella app \\
                      &     & dato un token di pernottamento valido \\ \hline
    
    \end{tabular}
    \caption{\label{administration-routes}Rotte della Rest API di Administration.}

\end{table}

\begin{figure}[H]
    \begin{center}
        \includegraphics[width=1.0\textwidth]{images/er-admin.png}
    \end{center}
    \caption{\label{er-admin}Modello ER dei dati di Administration}
\end{figure}

\subsection{AWS IoT Core}

Questo componente è delegato all'omonimo servizio Amazon, 
qui riportiamo solamente in figura \ref*{er-iot} la definizione della tabella per storicizzare i dati inviati delle centraline.

\begin{figure}[H]
    \begin{center}
        \includegraphics[width=0.3\textwidth]{images/er-iot.png}
    \end{center}
    \caption{\label{er-iot} Tabella per storicizzare i dati inviati dalle centraline}
\end{figure}


\subsection{Data Elaboration}

Il componente Data Elaboration comprende due servizi uno per calcolare e storicizzare i dati ed uno per fornirli con Rest API.
Per la prima funzionalità, non occorre una particolare architettura in quanto i calcoli possono avvenire in un'unica procedura avviata periodicamente.
Per la seconda parte è sufficiente un'unica rotta che richiede un token di accesso Cognito o fornito da Guest Authorization.

Maggiori dettagli per comprendere la progettazione di questo componente sono specificati nel capitolo 7 - deployment.

Riportiamo in figura \ref*{er-elaboration} la definizione della tabella per storicizzare i dati elaborati.

\begin{figure}[H]
    \begin{center}
        \includegraphics[width=0.3\textwidth]{images/er-elaboration.png}
    \end{center}
    \caption{\label{er-elaboration} Tabella per i dati elaborati}
\end{figure}

\newpage

\subsection{Guest Authorization}

Il componente Guest Authorization non richiede una particolare architettura in quanto è realizzabile con un unica procedura avviata su richiesta.
Per permettere ad Administration di avviare la funzione di questo componente ad ogni check-in/out è possibile adottare una Rest API con un'unica rotta 
oppure un meccanismo a messaggi: Administration invia un messaggio a Guest Authorization che reagisce lanciando la procedura.
Optiamo per la seconda possibilità come meglio specificato nel capitolo 7 - deployment.
