%----------------------------------------------------------------------------------------
%	INTRODUZIONE
%----------------------------------------------------------------------------------------

\section{Introduzione}

La sostenibilità in Europa è un tema molto sentito anche nel settore alberghiero, da anni infatti sono presenti sistemi e centraline che permettono di ridurre gli sprechi energetici andando ad esempio a spegnere gli impianti elettrici all'uscita dell'ospite dalla sua camera.

Negli USA invece la maggior parte degli hotel non dispone di questi semplici meccanismi e l'utenza è molto meno attenta al problema eco sostenibilità. Un banale esempio di spreco energetico che si presenta con un'elevata frequenza, è quello in cui l'ospite, specialmente d'estate, esce dalla camera lasciando il condizionatore acceso a temperature molto basse e contemporaneamente le tende aperte.

Dal 2020 la \href{https://sustainablehospitalityalliance.org/}{Sustainable Hospitality Alliance}, che comprende il 30\% dell'industria alberghiera globale, si impegna a proporre linee guida e \textit{best practice} per il design e la gestione di hotel sostenibili. Tuttavia il processo per la sostenibilità è molto lungo ed impegnativo e l'alleanza è ancora in una fase preliminare di raccolta dati.

Inoltre, senza immaginare l'arrivo di un evento dirompente come quello della guerra in Ucraina, il World Tourism Organization aveva divulgato un documento che riassumeva alcuni atteggiamenti utili a ridurre i consumi prodotti dal turismo. Secondo un rapporto della Commissione Europea (‘Consumption and Environment’), il settore economico del turismo si colloca al quarto posto pre fabbisogno energetico. Infine uno studio della Confesercenti rilasciato ad agosto 2022, prevede un aumento delle bollette per gli hotel di oltre 1.5 miliardi di euro nei prossimi 12 mesi.

L'obiettivo del nostro cliente è quello di proporre all'alleanza sopracitata Ecotrip, un nuovo sistema che faccia leva direttamente sugli ospiti promuovendone un comportamento virtuoso in termini di consumi energetici/acqua. 

%Il contributo tecnologico apportato dal progetto si riassume in una centralina (\textit{control unit}) installa in ogni stanza, una piattaforma in \textit{cloud} per la raccolta dei dati e un' applicazione lato smartphone che permette di mostrare all'ospite un "punteggio sostenibilità" che ne evidenza l'attitudine al corretto consumo delle risorse o viceversa allo spreco. La centralina consiste in un computer/micro-controllore che esegue un servizio sviluppato ad-hoc, il quale interroga periodicamente sensori dislocati nella stanza. Questi rilevano fattori ambientali come illuminazione e temperatura della stanza ma anche consumi elettrici/idrici.

Il contributo tecnologico apportato dal progetto si riassume in:
\begin{itemize}
    \item \textbf{centralina} (\textit{control unit}): dispositivo hardware, nello specifico un computer/micro-controllore installato in ogni stanza, che si occupa di rilevare periodicamente fattori ambientali e consumi elettrici/idrici tramite sensori opportunamente connessi. Questi sono stati programmati singolarmente avvalendosi dei manuali forniti dai produttori. Infine, si è realizzato un software ad-hoc che definisce la \textit{business logic} della centralina. Questo ad intervalli regolari interroga i sensori, acquisisce i dati e ne invia un aggregato ad una piattaforma in cloud. 
    \item \textbf{piattaforma in cloud}: si è scelto di usufruire dell'ecosistema di Amazon AWS per semplificare la gestione/mantenimento delle centraline (AWS IoT Core), mantenere in memoria le informazioni (AWS Cognito) ed implementare la formula del calcolo del punteggio tramite \textit{lambda}.
    \item \textbf{pannello di controllo}: applicazione \textit{frontend} utilizzata sia dagli amministratori di Ecotrip, sia dagli \textit{hotelier}. I primi si occupano dalla registrazione della registrazione di nuovi hotel, comprese le stanze, della creazione degli \textit{account} per gli albergatori e del monitoraggio delle centraline. I secondi invece usufruiscono del pannello al fine di registrare i pernottamenti.
    \item \textbf{app}: applicazione lato smartphone, consente all'ospite di visualizzare il proprio "punteggio sostenibilità" associato ad un determinato pernottamento, allo scopo di evidenziarne l'attitudine al corretto consumo delle risorse o viceversa allo spreco.
\end{itemize}
Il sistema Ecotrip si presenta come un valido prodotto data la possibilità di installarlo su hotel preesistenti senza aggravio di eccessivi costi di ristrutturazione. Inoltre, l'ottimizzazione dei consumi non si effettua solo tramite automazione ma necessità del contributo personale dell'utente. Questo deve essere sensibilizzato alla eco-sostenibilità ed qui che entra in gioco il concetto di "gamification" implementato dal sistema che diventa viatico alla riduzione degli sprechi.

\newpage