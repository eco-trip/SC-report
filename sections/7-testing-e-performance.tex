
%----------------------------------------------------------------------------------------
%	TESTING E PERFORMANCE
%----------------------------------------------------------------------------------------

\section{Testing e performance}

Lo sviluppo del progetto è progredito con metodologia incrementale e con test automatizzati per tutte le sue parti critiche. 
Ogni componente è quindi adeguatamente testato singolarmente nelle sue logiche di funzionamento.

A prototipo concluso si sono eseguiti manualmente test integrativi andando a verificare l'effettivo funzionamento del sistema nel suo insieme.
A tal fine, sono stati eseguiti i seguenti passi:
\begin{itemize}
    \item deployment dei diversi servizi web in abiente \textit{production}
    \item accesso al pannello di controllo con l'utente super amministratore pre configurato
    \item aggiunto da pannello di controllo un hotel di prova con il rispettivo account per l'albergatore
    \item aggiunta da pannello di controllo una camera all'hotel
    \item avviata la centralina
    \item installata e configurata da pannello IoT Core specificando hotel e camera come suoi attributi
    \item verificata la ricezione dei dati 
    \item verificata la correttezza dei dati in base alle sollecitazioni dei vari sensori
    \item eseguito da pannello di controllo un check-in per la camera generando il token
    \item verificato il calcolo del punteggio sostenibilità per il pernottamento di prova
    \item avviata la web app avvicinando lo smartphone al controller NFC
\end{itemize}

Non si sono riscontrati problemi tranne per l'invio del token al dispositivo cellulare mediante NFC: 
permane un problema che riguarda l'inizializzazione del controller NFC che a volte non avviene con successo. 
Abbiamo trovato un work-around ma il fix del problema richiede una sessione di debugging approfondita che riguarda 
il protocollo "NFC Forum Type 4 Tag".

Riguardo le performance del sistema, è possibile discutere di due componenti critiche: \textit{Control Unit} e \textit{Data Elaboration}.

Per la centralina si segnala che, per quanto l'implementazione in Java sia del tutto inefficiente e ci ha costretti ad eseguire campionamenti ripetuti su diversi sensori in modalità polling, 
l'utilizzo della CPU del Raspberry PI4 non supera il 12\% con una media intorno al 8\%.
Questo significa che, adottando anche un implementazione in C, possiamo ridurre di molto le caratteristiche dell'hardware e quindi risparmiare sui costi finali del prodotto.
Per quello che concerne l'invio dei dati con il protocollo MQTT, poichè questo avviene ogni 5 secondi ed il messaggio è di dimensione molto ridotta, non vi sono particolari analisi di performance da effettuare.

Per il componente \textit{Data Elaboration} la cui AWS Lambda si attiva periodicamente ogni 5 minuti abbiamo rilevato un tempo di esecuzione medio di 900 ms con utilizzo di RAM medio pari a 120 MB:
qui sono sicuramente possibili degli interventi volti a migliorarne le performance e ridurre i costi di servizio.

\newpage