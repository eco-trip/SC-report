%----------------------------------------------------------------------------------------
%	STATO DELL'ARTE
%----------------------------------------------------------------------------------------

\section{Stato dell'arte}

I sistemi di automazione per hotel sono diventati sempre più popolari negli ultimi anni come strumenti di promozione dell'efficienza energetica, di riduzione dei costi e di sostenibilità ambientale. Questi sistemi possono monitorare il consumo delle camere e identificare potenziali aree di spreco; le aziende che offrono questa gamma di soluzioni possono essere Siemens, Honeywell e Schneider Electric. Tra i loro prodotti vi sono:

\begin{itemize}
    \item \underline{sistema di monitoraggio}: consente ai manager degli hotel di monitorare e controllare il consumo di energia in tempo reale, tra cui illuminazione, riscaldamento e raffreddamento, portando a risparmi energetici fino al 30\%;
    \item \underline{termostato intelligente}: dispositivi che possono essere programmati per regolare automaticamente la temperatura delle camere in base all'occupazione e alle preferenze degli ospiti, portando a risparmi energetici fino al 10-15\%;
    \item \underline{sistema di automazione della camera}: consente agli ospiti di controllare diversi aspetti della loro camera, come illuminazione e temperatura, attraverso un'app mobile o un tablet in camera ai fini di comfort;
    \item \underline{prese intelligenti}: sono dispositivi che possono essere utilizzati per monitorare e controllare il consumo di energia di apparecchi specifici, come televisori e condizionatori d'aria, portando a risparmi energetici fino al 10-15\%;
    \item \underline{sistemi di gestione dell'edificio}: integrano più sistemi dell'edificio, tra cui HVAC, illuminazione e sicurezza, ai fini di efficienza energetica e comfort.
\end{itemize}

Un esempio concreto di sistemi di automazione per hotel è l'Hotel Mandarin Oriental di Bangkok che ha installato questo tipo di sistema, nello specifico di Siemens, per controllare l'illuminazione, il condizionamento dell'aria e il consumo di energia dell'hotel. Ciò ha portato a risparmi energetici del 30\% e a un risparmio annuale sui costi di 150.000 dollari. Un altro esempio è l'Hotel Le Meridien di Francoforte, che ha implementato termostati intelligenti in ogni camera degli ospiti, il che ha comportato una riduzione del 15\% del consumo di energia e un risparmio annuale sui costi di 90.000 dollari;

Come team, abbiamo sviluppato una soluzione unica per l'automazione degli hotel che mette al centro il cliente. Il nostro obiettivo è quello di monitorare il consumo e coinvolgere gli ospiti in comportamenti sostenibili attraverso l'utilizzo di tecniche di \textit{gamification} e premi. Ciò può essere realizzato senza la necessità di costi elevati e può essere installato anche in hotel datati evitando costi di ristrutturazione. Il nostro sistema premia gli ospiti per comportamenti sostenibili, per esempio tramite un sistema di punti o una classifica e potrebbe incorporare elementi di gioco, come sfide o \textit{quest}. Alcuni esempi di premi per gli ospiti potrebbero essere:
\begin{itemize}
    \item upgrade gratuito della camera;
    \item colazione gratuita;
    \item sconto sui futuri soggiorni;
    \item regalo \textit{eco-friendly} come una borraccia o una borsa da viaggio (\textit{tote bag}).
\end{itemize}
Ovviamente questi premi devono essere adattati al tipo/budget dell'hotel ed ai gusti del cliente. Questo approccio è vantaggioso sia per gli hotel poiché riduce il consumo di energia, quindi i costi, e aumenta la soddisfazione/fedeltà degli ospiti, sia per il cliente stesso poiché acquisisce consapevolezza sull'impatto delle proprie azioni in termini di eco-sostenibilità.
%
Molti hotel hanno già programmi di fedeltà o premi a punti che offrono vantaggi agli ospiti per soggiornare o prenotare attraverso specifici canali. Tuttavia, l'integrazione della tecnica di \textit{reward} per comportamenti virtuosi (sostenibili) è un concetto relativamente nuovo. Infatti alcuni hotel e associazioni si stanno muovendo verso questa direzione, per esempio:
\begin{itemize}
    \item La catena Kimpton Hotels ha avviato un programma chiamato \href{https://www.ihg.com/kimptonhotels/content/us/en/about-us/kimpton-cares/environment
    }{"EarthCare"}  che si concentra sulla riduzione dell'impatto ambientale attraverso la conservazione dell'energia/acqua, il riciclaggio/compostaggio e la promozione dell'approvvigionamento sostenibile; Offrono inoltre ai loro ospiti l'opzione di non usufruire dei servizi giornalieri di pulizia così da ricevere un credito di 5\$ al giorno da utilizzare al ristorante o al bar dell'hotel.
    \item Il programma \href{http://www.greenkeyglobal.com/home/green-key-eco-rating-2/
    }{Green Key Eco-Rating} premia gli hotel che raggiungono un certo livello di sostenibilità ambientale. Gli hotel che partecipano al programma vengono riconosciuti come "amici dell'ambiente" tramite un etichetta posta sia nel sito di Eco-Rating che in quello dell'hotel.
\end{itemize}
Queste però sono iniziative che non sfruttano le potenzialità di un sistema automatizzato e soprattutto non coinvolgono il cliente mediante lo strumento di \textit{gamification}.

\newpage