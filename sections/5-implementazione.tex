%----------------------------------------------------------------------------------------
%	IMPLEMENTAZIONE
%----------------------------------------------------------------------------------------

\section{Implementazione}\label{sec:implementazione}
Similmente a quanto fatto nella sezione precedente, anche in questo caso le soluzioni implementative avanzate sono state suddivise per sottodomonio. Di ognuna si elencano le motivazioni e le problematiche emerse durante lo sviluppo. 

\subsection{Control Unit}
Per quanto riguarda la centralina, prima di avviare il vero e proprio sviluppo del progetto si è scelto di verificare il corretto funzionamento dei sensori, collegando ciascuno di essi alla scheda e validando l'output restituito tramite opportuni software rilasciati dal produttore. Nello specifico, siccome diversi sensori comunicano mediante protocollo I2C, è stato necessario collegare questi a una \textit{breadboard} al fine di ampliare il numero di ingressi disponibili ai GPIO SDA e SCL. Una volta completata la fase preliminare di sperimentazione si è passati alla realizzazione di una \textit{build multi-progetto} confinando le funzionalità all'interno di moduli indipendenti, al fine di semplificare l'attività di sviluppo parallela tra i membri del team. Come già anticipato nella sezione di progettazione, il progetto si compone di diversi strati, quelli più interni realizzano la logica di \textit{business} che deve essere totalmente indipendente da \textit{framework}. I livelli più esterni invece dipendo dalla libreria \href{https://pi4j.com/}{Pi4J} (versione 2), la motivazione principale è stata la ricerca di un livello di astrazione che potesse nascondere gli aspetti di basso livello, permettendo così al team di concentrasi quasi esclusivamente sulla logica di \textit{business}. Ovviamente vi sono altri vantaggi derivanti dall'adozione di Pi4J e possono essere riassunti in:
\begin{itemize}
    \item \underline{Facilità d'uso}: Pi4J espone un'API semplice e intuitiva in Java che facilita l'interazione con la componente hardware, cioè con i pin GPIO del Raspberry;
    \item \underline{Supporto per più periferiche}: Pi4J supporta anche altri dispositivi periferici connessi al Raspberry Pi come dispositivi I2C e SPI;
    \item \underline{Manutenzione}: Pi4J è attivamente mantenuta da una comunità di sviluppatori, questo comporta la risoluzione di eventuali bug ed estensioni future;
    \item \underline{Grande comunità}: in rete vi sono numerose risorse (es. classi Java che modellano il comportamento di specifici sensori) e un ampio supporto per la risoluzione dei problemi;
    \item \underline{Open-source}.
\end{itemize}
Il \textit{framework} Pi4J, da una parte ha agevolato lo sviluppo architetturale ma dall'altra ha reso gli \textit{script} forniti dai produttori inutilizzabili, vincolando quindi il team a riscrivere le logiche dei singoli sensori. Le classi prodotte sono quindi il risultato di una ricerca basata sugli \textit{script} dei fornitori, i \textit{datasheet} dei sensori e le risorse in rete messe a disposizione dalla \textit{community}.
\newline\newline
Siccome il software della centralina si compone di due servizi, cioè \textit{Authorization} e \textit{RoomMonitoring}, fin da subito si è optato per l'adozione del paradigma concorrente così da eseguire i due servizi in flussi di controllo separati. Nella pratica si è realizzata una classe \texttt{Engine} che aggiunge un livello di astrazione rispetto allo \texttt{ScheduledExecutorService} della libreria \texttt{java.util.concurrent}. Concretamente i servizi vengono lanciati nel seguente modo:
\lstinputlisting[caption={Setup e avvio dei servizi.}, label={lst:main}, language=Java]{code/main.java}
%
Come mostrato nel Listato \ref{lst:main}, per ogni servizio viene istanziato un \texttt{Engine} fatta eccezione per l'\texttt{awsAdapter} che, appoggiandosi sulla libreria \texttt{software.amazon.awssdk}, già esegue all'interno di un flusso di controllo indipendente.L'\texttt{awsAdapter} si occupa di inoltrare i dati prodotti dal \texttt{RoomMonitoringService} verso il \textit{backend} tramite protocollo MQTT e realizza lo \textit{shadowing} tra la centralina e la sua copia digitale. 
Inoltre l'\texttt{awsAdapter} registra l'\texttt{AuthorizationService} come "osservatore" così da notificargli eventuali modifiche del \textit{token} di accesso. Infine, l'avvio dei servizi avviene solamente conclusa la procedura di \textit{bootstrap}, cioè di connessione con il \textit{backend}.
%
\subsubsection{RoomMonitoringService}
Internamente il \texttt{RoomMonitoringService} (Listato \ref*{lst:rms}) esegue ogni $N$ secondi il calcolo dei consumi (corrente e acqua divisa in calda e fredda) e rileva i fattori ambientali della stanza (temperatura, umidità, luminosità). I dati raccolti vengono prima serializzati e poi inviati verso l'esterno. I consumi vengono calcolati più volte durante gli $N$ secondi, infatti è prevista una logica di aggregazione. Tutte le funzionalità implementante sono asincrone poiché si avvalgono delle \texttt{CompletableFuture} di Java.
\lstinputlisting[caption={Logica \textit{core} del \texttt{RoomMonitoringService}.}, label={lst:rms}, language=Java]{code/RoomMonitoringService.java}
%
\subsubsection{AuthorizationService}
L'\texttt{AuthorizationService} esegue solo un \textit{task} bloccante definito in due step: \texttt{bootstrap()} e \texttt{startNfcTagEmulation()}. Il primo provvede a sincronizzare lo stato della centralina con quello memorizzato sul \textit{backend}, mentre il secondo abilita la connessione tra lo smartphone e il lettore NFC. Come mostrato nel Listato \ref*{lst:as}, la funzione \texttt{startNfcTagEmulation} viene ripetuta anche nel caso di due errori specifici. Quest'ultimi non sono considerati fatali poiché possono essere provocati, per esempio da un allontamento precoce dello smartphone dal lettore.
\lstinputlisting[caption={Logica \textit{core} dell'\texttt{AuthorizationService}.}, label={lst:as}, language=Java]{code/AuthorizationService.java}
%
Il codice del programma che permette di convertire i dati ottenuti dal sensore BH1750 in \texttt{lux} (Listato \ref*{lst:bh}), si avvale della classe \texttt{I2C} di Pi4j che, grazie al metodo \texttt{read}, restituisce i due byte (LSB e MSB) necessari per produrre il risultato.
\lstinputlisting[caption={Progammazione del modulo BH1750.}, label={lst:bh}, language=Java]{code/BH1750.java}
%
\subsubsection{DHT22}
Nel caso del modulo DHT22, la programmazione del sensore ha richiesto uno sforzo maggiore poiché si è riscontrata una problematica legata all'interfaccia \texttt{Digital} fornita dalla libreria Pi4J. Questa permette di comunicare con un pin GPIO del Raspberry mediante due classi: \texttt{DigitalOutput} e \texttt{DigitalInput}. Quindi non è possibile creare un collegamento sul quale scrivere e leggere contemporaneamente. Questo si è rivelato una limitazione perché il protocollo del DHT22 richiede nello stesso ciclo di polling, una comunicazione sia in input che in output, ma il tempo richiesto per la riconfigurazione del pin provoca un ritardo che quasi sempre supera la finestra di 40-80 microsecondi richiesta dal protocollo. Ciò comporta una perdita di informazioni che può provocare errori nella trasmissione. Si è quindi scelto di realizzare un \textit{fork} di Pi4J al fine di aggiungere le funzionalità richieste all'interno di una nuova classe chiamata \texttt{DigitalMultiPurpose}. Questa implementazione permette di interagire con un pin in entrambe le modalità riducendo i tempi allo scopo di rientrare all'interno della finestra temporale prima citata. Della libreria Pi4J è stato modificato anche il file di \textit{build} di \textit{Maven} così da permettere la pubblicazione degli artefatti (\textit{package}) all'interno di Github Packages automatizzandone il download tramite Gradle.
%
\subsubsection{Flussometro}
Per quanto riguarda il flussometro, è stato sufficiente utilizzare un istanza di \texttt{DigitalInput} messe a disposizione da Pi4J. A questa è stata associata una \textit{callback} che si occupa di incrementare un contatore ogni qualvolta il pin si trovi nello stato di \texttt{HIGH}. Dopo un secondo (circa) il contatore degli impulsi viene suddiviso per la frequenza specifica del sensore così da ottenere i litri di acqua fluiti (L/s). Al fine di ottenere un risultato più accurato, si tiene traccia del tempo effettivamente trascorso da una rilevazione e l'altra.
\lstinputlisting[caption={Progammazione del flussometro.}, label={lst:wfs}, language=Java]{code/WaterFlowSensor.java}
%
\subsubsection{NTC - Termistore}
Come già anticipato nella sezione progettuale, per il calcolo della temperatura tramite termistore ci si affida ad una versione semplificata dell'equazione di Steinhart-Hart. L'utilizzo della costante \texttt{KELVIN} (273.15) permette di convertire il risultato ottenuto in Celsius. Poiché il termistore utilizzato è un dispositivo analogico, il parametro \texttt{inputVoltage} viene fornito dall'ADC a sua volta astratto mediante una specifica classe Java.
\lstinputlisting[caption={Progammazione del termistore NTC 50K.}, label={lst:ntc}, language=Java]{code/NTCSensor.java}
% 
\subsubsection{PN532}

Il modulo NFC nel nostro caso non si occupa di leggere TAG, ma di stabilire una comunicazione con uno smartphone al fine di inviargli il token.
Più precisamente si è riusciti ad eseguire questa operazione senza un app nativa sul dispositvo cellulare: 
è infatti sufficiente avvicinare lo smartphone al controller NFC perchè questo si attivi ed invii URL con il token lanciando la web app sul dispositivo.

Per fare questo il controller NFC viene inizalizzato via software nella modalità "card emulation", il PN532 è quindi capace di rispondere a comandi "Reader/Writer" 
secondo lo schema standard ISO/IEC 14443A/MIFARE.

Nell'implementare questa modalità di comunicazione si sono seguite le specifiche di "NFC Forum Type 4 Tag".
Un Tag Type 4 contiene un messaggio NFC Data Exchange Format (NDEF) il quale include il nostro URL con il token.

Il controller PN532 ci ha dato dapprima problemi di stabilità di connessione con l'interfaccia I2C, che ci ha costretti a connetterlo al canale SPI.
In secondo luogo abbiamo dovuto riportare tutto il protocollo di comunicazione in linguaggio Java e questo ha richiesto una gran quantità di debugging.

Attualmente il controller funziona solo comunicando con dispositivo modello iPhone, occorre più tempo per supportare altri device e rendere tutto più stabile. 

\newpage