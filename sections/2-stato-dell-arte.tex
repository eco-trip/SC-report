%----------------------------------------------------------------------------------------
%	STATO DELL'ARTE
%----------------------------------------------------------------------------------------

\section{Stato dell'arte}

I sistemi di automazione e monitoraggio per hotel sono diventati sempre più popolari negli ultimi anni come strumenti di promozione dell'efficienza energetica, di riduzione dei costi e di sostenibilità ambientale. Questi sistemi possono monitorare il consumo delle camere e identificare potenziali aree di spreco; le aziende che offrono questa gamma di soluzioni possono essere Siemens, Honeywell e Schneider Electric. Tra i loro prodotti vi sono:

\begin{itemize}
    \item \underline{sistemi di monitoraggio}: consentono ai manager degli hotel di monitorare e controllare il consumo di energia in tempo reale, tra cui illuminazione, riscaldamento e raffreddamento;
    \item \underline{termostati intelligenti}: dispositivi che possono essere programmati per regolare automaticamente la temperatura delle camere in base all'occupazione e alle preferenze degli ospiti;
    \item \underline{sistemi di automazione della camera}: consentono agli ospiti di controllare diversi aspetti della loro camera, come illuminazione e temperatura, attraverso un'app mobile o un tablet in camera ai fini di comfort;
    \item \underline{prese intelligenti}: dispositivi che possono essere utilizzati per monitorare e controllare il consumo di energia di apparecchi specifici, come televisori e condizionatori d'aria;
    \item \underline{sistemi di gestione dell'edificio}: integrano più sistemi dell'edificio, tra cui HVAC, illuminazione e sicurezza, ai fini di efficienza energetica e comfort.
\end{itemize}

Hilton Hotels \& Resorts ha avviato nel 2009 un programma \href{https://stories.hilton.com/hilton-history/lightstay-a-decade-of-managing-our-environmental-and-social-impact}{"LightStay"}, che utilizza la tecnologia per monitorare il consumo energetico, i rifiuti e l'utilizzo dell'acqua dell'hotel. Questo sistema ha aiutato Hilton Hotels \& Resorts a ridurre l'impatto ambientale e i costi. Secondo i dati pubblicati dalla compagnia, dal 2009 al 2019 le proprietà Hilton hanno ridotto le emissioni di carbonio equivalenti alla rimozione di 390.350 auto dalla strada e hanno contribuito 6.273.934 ore di volontariato nelle loro comunità locali, risparmiando oltre 1 miliardo di dollari in costi di utilità. Un altro \href{https://new.abb.com/news/detail/87127/abbs-hvac-solution-helps-jw-marriott-hotel-to-cut-down-energy-losses-by-35}{esempio concreto}, è il sistema di gestione dell'edificio implementato dal JW Marriott Hotel di Los Angeles, il quale ha ottimizzato i consumi energetici riducendo gli sprechi del 35\%. 

Come team, si è definita una soluzione unica per il monitoraggio energetico degli hotel che mette al centro il cliente. L'obiettivo è quello di rilevare il consumo e di coinvolgere gli ospiti in comportamenti sostenibili attraverso l'utilizzo di tecniche di \textit{gamification} e premi. Il sistema premia gli ospiti per comportamenti sostenibili, per esempio tramite un sistema di punti o una classifica e potrebbe incorporare elementi di gioco, come sfide o \textit{quest}. Alcuni esempi di premi per gli ospiti potrebbero essere:
\begin{itemize}
    \item upgrade gratuito della camera;
    \item colazione gratuita;
    \item sconto sui futuri soggiorni;
    \item regalo \textit{eco-friendly} come una borraccia o una borsa da viaggio (\textit{tote bag}).
\end{itemize}
Ovviamente questi premi devono essere adattati al tipo/budget dell'hotel ed ai gusti del cliente. Questo approccio è vantaggioso sia per gli hotel, poiché riduce il consumo di energia e aumenta la soddisfazione/fedeltà degli ospiti, sia per il cliente stesso dato che acquisisce consapevolezza sull'impatto delle proprie azioni in termini di eco-sostenibilità.
%
Molti hotel hanno già programmi di fedeltà o premi a punti, tuttavia l'integrazione della tecnica di \textit{reward} per comportamenti virtuosi (sostenibili) è un concetto relativamente nuovo. Infatti alcuni hotel e associazioni si stanno muovendo verso questa direzione, per esempio:
\begin{itemize}
    \item La catena Kimpton Hotels ha avviato un programma chiamato \href{https://www.ihg.com/kimptonhotels/content/us/en/about-us/kimpton-cares/environment
    }{"EarthCare"}  che si concentra sulla riduzione dell'impatto ambientale attraverso la conservazione dell'energia/acqua, il riciclaggio/compostaggio e la promozione dell'approvvigionamento sostenibile; Offrono inoltre ai loro ospiti l'opzione di non usufruire dei servizi giornalieri di pulizia così da ricevere un credito di 5\$ al giorno da utilizzare al ristorante o al bar dell'hotel.
    \item Il programma \href{http://www.greenkeyglobal.com/home/green-key-eco-rating-2/
    }{Green Key Eco-Rating} premia gli hotel che raggiungono un certo livello di sostenibilità ambientale. Gli hotel che partecipano al programma vengono riconosciuti come "amici dell'ambiente" tramite un etichetta posta sia nel sito di Eco-Rating che in quello dell'hotel.
    \item L'InterContinental Hotels Group ha un programma chiamato \href{https://www.ihg.com/content/us/en/about/green-engage}{"IHG Green Engage"} che premia gli ospiti attraverso punti in relazione alla scelta di risparmiare energia e risorse. Questo programma consente inoltre all'hotel di monitorare le proprie prestazioni ambientali e di fissare obiettivi di miglioramento.
\end{itemize}
Queste però sono iniziative che non sfruttano le potenzialità derivanti dall'adozione di un sistema digitalizzato, in genere si basano su scelte consapevoli degli ospiti. La novità della soluzione proposta è proprio quella di trovare un punto di intersezione tra i programmi/iniziative sostenibili e la tecnologia di cui il meccanismo abilitante è l'IoT combinato con la \textit{gamification}. Rispetto ai sistemi disponibili sul mercato, Ecotrip può essere installato su un qualsiasi hotel, anche se datato, a un prezzo calmierato poiché si evitano i costi di ristrutturazione.

\newpage
