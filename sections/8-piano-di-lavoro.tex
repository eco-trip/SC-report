%----------------------------------------------------------------------------------------
%	PIANO DI LAVORO
%----------------------------------------------------------------------------------------

\section{Piano di lavoro}

Lo sviluppo del progetto Ecotrip ha coinvolto il team per uno span temporale relativamente lungo di circa 8 mesi, 
in quanto è stato inserito in mezzo a impegni di lavoro.
Indicativamente ogni componente del team ha svolto più di 1 mese e mezzo di attività di sviluppo per giungere al prototipo finale, 
questo è andato oltre le aspettative iniziali ma con la consapevolezza che l'impegno potrebbe portare ad una startup 
con il nostro partner commerciale.

La prima fase del progetto è stata svolta dall'intero team per circa 1 settimana nel mese di Giugno 2022, ed ha compreso:
\begin{itemize}
    \item ideazione e knowledge crunching con il partner
    \item studio preliminare di fattibilità
    \item progettazione strategica e architetturale
    \item stesura dei requisiti
    \item impostazione e creazione dei repo di lavoro
\end{itemize}

Si è poi proceduto con lo sviluppare singoli sezioni del progetto, da qui ogni componente del team ha lavorato singolarmente o al massimo in coppia.

Di seguito illustreremo per ogni componente del team ciò che è stato fatto in ordine cronologico.

Alan Mancini
\begin{itemize}
    \item Control Unit: ricerca, selezione e acquisizione dei sensori (2gg - Giugno 2022)
    \item Control Unit: studio ed integrazione framework PI4J (2gg - Giugno 2022) 
    \item Control Unit: installazione e sperimentazione dei vari sensori (7gg - Giugno 2022)
    \item Control Unit: implementazione architettura esagonale e logica applicativo (7gg - Settembre 2022)
    \item Control Unit: fork e miglioramenti a PI4J (7gg - Novembre 2022)
    \item Control Unit: programmazione sensore ACS712 (2gg - Novembre 2022)
    \item IoT Core: studio DynamoDB (1gg - Novembre 2022)
    \item IoT Core: studio e configurazione regole storicizzazione (1gg - Novembre 2022)
    \item Control Unit: studio AWS IoT Device per stato shadow (1gg - Dicembre 2022)
    \item Control Unit: programmazione NFC PN532 (8gg - Dicembre 2022)
    \item Guest Authorization: studio AWS Lambda con SQS e implementazione (5gg - Gennaio 2023)
\end{itemize}

Alberto Marfoglia
\begin{itemize}
    \item Control Unit: impostazione progetto Gradle e Continuous Integration (2gg - Giugno 2022) 
    \item Control Unit: ricerca framework per Raspberry (1gg - Giugno 2022) 
    \item Control Unit: installazione e sperimentazione dei vari sensori (7gg - Giugno 2022)
    \item Control Unit: implementazione architettura esagonale e logica applicativo (14gg - Settembre 2022)
    \item Control Unit: programmazione sensore DHT22 (3gg - Novembre 2022)
    \item Control Unit: programmazione sensore CFB7 (2gg - Novembre 2022)
    \item Control Unit: programmazione sensore BH1750 (1gg - Novembre 2022)
    \item IoT Core: studio e setup (1gg - Dicembre 2022)
    \item Control Unit: studio AWS SDK e MQTT (2gg - Dicembre 2022)
    \item Control Unit: implementazione AWS Adapter (4gg - Dicembre 2022)
    \item Guest Authorization: studio AWS Lambda con SQS e implementazione (5gg - Gennaio 2023)
    \item Data Elaboration: studio AWS API Gateway per Rest API serverless (2gg - Febbraio 2023)
\end{itemize}

Matteo Brocca
\begin{itemize}
    \item AWS: studio Cognito per autenticazione e regole di accesso ai servizi (1gg - Giugno 2022) 
    \item AWS: studio e sperimentazione SEM/cloudformation (3gg - Giugno 2022) 
    \item AWS: configurazione di deployment SEM dei vari servizi (8gg - Giugno 2022) 
    \item Control Unit: implementazione architettura esagonale e logica applicativo (4gg - Settembre 2022)
    \item Control Unit: programmazione convertitore ADS1115 (1gg - Novembre 2022)
    \item Control Unit: programmazione sensore NTC (1gg - Novembre 2022)
    \item Administration: studio DynamoDB (1gg - Dicembre 2022)
    \item Administration: implementazione Rest api con NodeJS (7gg - Dicembre 2022) 
    \item Control Panel: implementazione frontend React (4gg - Gennaio 2023) 
    \item Data Elaboration: studio AWS Lambda con Event Bus e implementazione (3gg - Febbraio 2023)
    \item Data Elaboration: studio AWS API Gateway ed impleemntazione Rest API serverless (3gg - Febbraio 2022)
    \item App: implementazione frontend React (3gg - Febbraio 2023) 
    \item App: studio MQTT per dati in tempo reale (3gg - Febbraio 2023) 
\end{itemize}
