%----------------------------------------------------------------------------------------
%	PIANO DI LAVORO
%----------------------------------------------------------------------------------------

\section{Piano di lavoro}

Lo sviluppo del progetto Ecotrip ha coinvolto il team per uno span temporale relativamente lungo di circa 8 mesi, 
in quanto è stato inserito in mezzo a impegni di lavoro.
Indicativamente ogni componente del team ha svolto più di 1 mese e mezzo di attività di sviluppo per giungere al prototipo finale, 
questo è andato oltre le aspettative iniziali ma con la consapevolezza che l'impegno potrebbe portare ad una startup 
con il nostro partner commerciale.

La prima fase del progetto è stata svolta dall'intero team per circa 1 settimana nel mese di Giugno 2022, ed ha compreso:
\begin{itemize}
    \item ideazione e knowledge crunching con il partner
    \item studio preliminare di fattibilità
    \item progettazione strategica e architetturale
    \item stesura dei requisiti
    \item impostazione e creazione dei repo di lavoro
\end{itemize}

Si è poi proceduto con lo sviluppare singoli sezioni del progetto, da qui ogni componente del team ha lavorato singolarmente o al massimo in coppia.

Di seguito illustreremo per ogni componente ciò che è stato fatto in ordine cronologico.

Alan Mancini
\begin{itemize}
    \item Control Unit: ricerca, selezione e acquisizione dei sensori (2gg - Giugno 2022)
    \item Control Unit: studio ed integrazione framework PI4J (2gg - Giugno 2022) 
    \item Control Unit: sperimentazione dei vari sensori (4gg - Giugno 2022)
    \item Control Unit: implementazione clean architecture (3gg - Settembre 2022)
    \item Control Unit: implementazione sensore ---- (1gg - Luglio 2022)
    \item Control Unit: implementazione sensore ---- (1gg - Luglio 2022)
    \item IoT Core: studio e configurazione regole storicizzazione (1gg - Luglio 2022)
\end{itemize}

Alberto Marfoglia
\begin{itemize}
    \item Control Unit: impostazione progetto Gradle e Continuous Integration (1gg - Giugno 2022) 
    \item Control Unit: ricerca framework per Raspberry (1gg - Giugno 2022) 
    \item Control Unit: sperimentazione dei vari sensori (4gg - Giugno 2022)
    \item Control Unit: implementazione clean architecture (7gg - Settembre 2022)
    \item Control Unit: implementazione sensore ---- (1gg - Novembre 2022)
    \item Control Unit: implementazione sensore ---- (1gg - Novembre 2022)
    \item IoT Core: studio e setup (1gg - Luglio 2022)
\end{itemize}

Matteo Brocca
\begin{itemize}
    \item AWS: studio e sperimentazione SEM/cloudformation (3gg - Giugno 2022) 
    \item AWS: configurazione di deployment SEM dei vari servizi (5gg - Giugno 2022) 
    \item Control Unit: implementazione clean architecture (1gg - Settembre 2022)
    \item Control Unit: implementazione sensore ---- (1gg - Dicembre 2022)
    \item Administration: implementazione rest api (5gg - Dicembre 2022) 
    \item Control Panel: implementazione frontend (3gg - Dicembre 2022) 
    
\end{itemize}






In questa sezione devono essere chiariti i compiti svolti da ciascun candidato nel caso in cui il gruppo abbia più di un componente.\\

Deve essere inoltre esposto il piano di lavoro adottato. A tal fine, per ogni attività svolta durante la preparazione dell'elaborato (ad esempio: studio di una tecnologia, progettazione di un componente, implementazione di un algoritmo ecc…) deve essere chiarita la collocazione temporale e devono essere indicate le risorse impiegate per svolgerla (giorni/uomo). I candidati possono ricorrere a opportuni diagrammi come quello di Gantt.\\


Vincoli circa la lunghezza della sezione (escluse didascalie, tabelle, testo nelle immagini, schemi):

\vspace{1cm}
\begin{tabular}{l|rr}
 & Numero minimo di battute & Numero massimo di battute \\
 \hline
 1 componente & 1000 & 2000 \\
 2 componenti & 1500 & 3000 \\
 3 componenti & 2000 & 4000 \\
 \hline
\end{tabular}

\newpage