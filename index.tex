\documentclass[12pt,a4paper]{article}
\usepackage[a4paper, total={6in, 9in}]{geometry}
\usepackage[italian]{babel}
\usepackage{newlfont}
\usepackage{float}
\usepackage{wrapfig}
\usepackage{natbib}
\usepackage{graphicx}
\usepackage{listings}
\usepackage{xcolor}
\usepackage{color}
\usepackage{hyperref}
\usepackage{titling}
\usepackage{caption}

\captionsetup[figure]{font=small,labelfont=bf,justification=centering,margin=1.5cm}

\newcommand{\emailaddr}[1]{\href{mailto:#1}{\texttt{#1}}}

\newcommand{\subtitle}[1]{%
  \posttitle{%
    \par\end{center}
    \begin{center}\Large#1\end{center}
    \vskip0.5em}%
}

\renewcommand{\labelenumii}{\arabic{enumi}.\arabic{enumii}}
\renewcommand{\labelenumiii}{\arabic{enumi}.\arabic{enumii}.\arabic{enumiii}}
\renewcommand{\labelenumiv}{\arabic{enumi}.\arabic{enumii}.\arabic{enumiii}.\arabic{enumiv}}

\definecolor{mygray}{rgb}{0.5,0.5,0.5}
\definecolor{mydarkgray}{rgb}{0.4,0.4,0.4}
\definecolor{mylightgray}{rgb}{0.9,0.9,0.9}
\definecolor{mygreen}{rgb}{0.0,0.45,0.0}

\lstdefinestyle{bash}{
  numbers=none,  
  keywords={npm,cd},
  keywordstyle=\color{mygreen}\bfseries,
  comment=[l]{\#},
  commentstyle=\color{mydarkgray}
}

\lstdefinestyle{java}{
	keywords={typeof, new, true, false, catch, function, return, null, catch, switch, var, if, in, while, do, else, case, break},
  keywordstyle=\color{blue}\bfseries,
  ndkeywords={class, export, boolean, throw, implements, import, this},
  ndkeywordstyle=\color{darkgray}\bfseries,
  identifierstyle=\color{black},
  comment=[l]{//},
  morecomment=[s]{/*}{*/},
  stringstyle=\color{red}\ttfamily,
  morestring=[b]'
}

\lstset{
  basicstyle=\footnotesize\ttfamily,
  backgroundcolor=\color{mylightgray},
  frame=none,
  tabsize=2,
  aboveskip=1em,
	belowskip=1em,
  sensitive=false,
  breaklines=true,
}

\hypersetup{
  colorlinks=true,
  linkcolor=black,
  citecolor=black,
  urlcolor=black,
  filecolor=black      
}

\graphicspath{ {img/} }

\textwidth=450pt\oddsidemargin=0pt

\begin{document}

\title{\LARGE \bf
    Ecotrip
}

\subtitle{Incentivo alla sostenibilità nel settore ospitality tramite gamification}

\author{
    Alan Mancini \\ \emailaddr{alan.mancini@studio.unibo.it}
    \and 
    Matteo Brocca \\ \emailaddr{matteo.brocca@studio.unibo.it} 
    \and 
    Alberto Marfoglia \\ \emailaddr{alberto.marfoglia@studio.unibo.it} 
}

\date{Maggio 2022}

\maketitle

\begin{abstract}
  Nel settore alberghiero il tema della sostenibilità sta acquisendo una certa importanza anche fuori dall'Europa, 
  dove spesso le camere d'albergo non sono nemmeno dotate di centraline automatizzate che riducono i consumi energetici quando l'ospite non è presente.

  Si vuole proporre Ecotrip, un sistema che incentivi la sostenibilità stimolando direttamente gli ospiti a un corretto comportamento grazie alla gamification.
  Il sistema è composto da una centralina, da installare nella camera d'albergo, che attraverso diversi sensori rileva i consumi di energia ed acqua e come questi vengono impiegati.
  Gli ospiti attraverso un'apposita applicazione su smartphone ottengono un punteggio in base al comportamento tenuto durante il proprio periodo di pernottamento, il punteggio 
  può essere convertito in uno sconto al checkout.

  In particolare la centralina misura i consumi di energia elettrica relativi al circuito luci e al circuito prese, il flusso di acqua fredda e calda con relative temperature,
  la temperatura e umidità d'ambiente ed infine è in grado di determinare se le tende sono aperte o chiuse con sensori di luminosità ambientale.
  I dati vengono campionati a intervalli regolari e inviati sul cloud.

  Il gestore dell'albergo dispone di un pannello di controllo per monitorare i dati delle varie stanze.
  Gli ospiti si possono collegare con il proprio smartphone alla stanza leggendo un tag NFC.


  Gli ospiti si possono collegare alla propria stanza al primo ingresso, attraverso un tag NFC installato a muro. 
  Il tag è collegato alla centralina che è in grado di riprogrammarlo quotidianamente in base ai pernottamenti, 
  in modo tale che un ospite dopo il checkout non abbia più accesso ai dati.
\end{abstract}

\newpage

\tableofcontents
\newpage

\include{sections/intro} 

\bibliographystyle{plain}
\bibliography{references}

\end{document}