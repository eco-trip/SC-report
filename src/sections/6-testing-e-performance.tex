
%----------------------------------------------------------------------------------------
%	TESTING E PERFORMANCE
%----------------------------------------------------------------------------------------

\section{Testing e performance}

Esporre lo stato di funzionamento effettivo del sistema progettato ad elaborato concluso. Per ciascuna delle funzionalità salienti devono essere tabellate e discusse le performance riscontrate mediante opportuni test eseguiti in fase di validazione del progetto.\\

I tempi di esecuzione/comunicazione devono essere accompagnati dalle caratteristiche dell'hardware sul quale è eseguito il software.\\

Qualora l'elaborato includa algoritmi innovativi, indicarne la complessità computazionale (avendo cura di esporre lo pseudo codice nella sezione \ref{sec:implementazione}).\\


Vincoli circa la lunghezza della sezione (escluse didascalie, tabelle, testo nelle immagini, schemi):

\vspace{1cm}
\begin{tabular}{l|rr}
 & Numero minimo di battute & Numero massimo di battute \\
 \hline
 1 componente & 2000 & 3000 \\
 2 componenti & 2500 & 4500 \\
 3 componenti & 3000 & 6000 \\
 \hline
\end{tabular}


\newpage