
%----------------------------------------------------------------------------------------
%	TESTING E PERFORMANCE
%----------------------------------------------------------------------------------------

\section{Testing e performance}

Lo sviluppo del progetto è progredito con metodologia incrementale e con test automatizzati per tutte le sue parti critiche. 
Ogni componente è quindi adeguatamente testato singolarmente nelle sue logiche di funzionamento.

A prototipo concluso si sono eseguiti manualmente test integrativi andando a verificare l'effettivo funzionamento del sistema nel suo insieme.
A tal fine, sono stati eseguiti i seguenti passi:
\begin{itemize}
    \item deployment dei diversi servizi web
    \item aggiunto da pannello di controllo un hotel di prova con il rispettivo account per l'albergatore
    \item aggiunta da pannello di controllo una camera all'hotel
    \item avviata la centralina
    \item installata e configurata da pannello IoT Core specificando hotel e camera
    \item verificata la ricezione dei dati 
    \item verificata la correttezza dei dati in base alle sollecitazioni dei vari sensori
    \item eseguito da pannello di controllo un check-in per la camera generando il token
    \item verificato il calcolo del punteggio sostenibilità per il pernottamento di prova
    \item avviata la web app avvicinando lo smartphone al controller NFC
\end{itemize}

Non si sono riscontrati problemi tranne per l'invio del token al dispositivo cellulare mediante NFC: 
permane un problema che riguarda l'inizializzazione del controller NFC che a volte non avviene con successo. 
Abbiamo trovato un work-around ma il fix del problema richiede una sessione di debugging approfondita che riguarda il protocollo "NFC Forum Type 4 Tag".

Riguardo le performance del sistema, è possibile discutere di due componenti: centralina e data elaboration.
Per la centralina si segnala che : java fa schifo ma la CPU non è usata. I dati inviati sono ridicoli.
Per il data elaboration la lambda sta attiva X secondi.


Per quanto concerne il resto dei servizi web non ci sono particolari osservazioni da fare.

\newpage