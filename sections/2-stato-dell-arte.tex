%----------------------------------------------------------------------------------------
%	STATO DELL'ARTE
%----------------------------------------------------------------------------------------

\section{Stato dell'arte}

I sistemi di automazione per hotel sono diventati sempre più popolari negli ultimi anni come strumenti di promozione dell'efficienza energetica, di riduzione dei costi e di sostenibilità ambientale. Questi sistemi possono monitorare il consumo delle camere e identificare potenziali aree di spreco; le aziende che offrono questa gamma di soluzioni possono essere Siemens, Honeywell e Schneider Electric. Tra i loro prodotti vi sono:

\begin{itemize}
    \item \underline{sistemi di monitoraggio}: consentono ai manager degli hotel di monitorare e controllare il consumo di energia in tempo reale, tra cui illuminazione, riscaldamento e raffreddamento, portando a risparmi energetici fino al 30\%;
    \item \underline{termostati intelligenti}: dispositivi che possono essere programmati per regolare automaticamente la temperatura delle camere in base all'occupazione e alle preferenze degli ospiti, portando a risparmi energetici fino al 10-15\%;
    \item \underline{sistemi di automazione della camera}: consentono agli ospiti di controllare diversi aspetti della loro camera, come illuminazione e temperatura, attraverso un'app mobile o un tablet in camera ai fini di comfort;
    \item \underline{prese intelligenti}: dispositivi che possono essere utilizzati per monitorare e controllare il consumo di energia di apparecchi specifici, come televisori e condizionatori d'aria, portando a risparmi energetici fino al 10-15\%;
    \item \underline{sistemi di gestione dell'edificio}: integrano più sistemi dell'edificio, tra cui HVAC, illuminazione e sicurezza, ai fini di efficienza energetica e comfort.
\end{itemize}

Un esempio concreto di sistemi di automazione per hotel è l'Hotel Mandarin Oriental di Bangkok che ha installato questo tipo di sistema, nello specifico di Siemens, per controllare l'illuminazione, il condizionamento dell'aria e il consumo di energia dell'hotel. Ciò ha portato a risparmi energetici del 30\% e a un risparmio annuale sui costi di 150.000 dollari. Un altro esempio è l'Hotel Le Meridien di Francoforte, che ha implementato termostati intelligenti in ogni camera degli ospiti, il che ha comportato una riduzione del 15\% del consumo di energia e un risparmio annuale sui costi di 90.000 dollari;

%    The InterContinental Hotels Group (IHG) has implemented energy management systems in many of its hotels worldwide. These systems monitor and control energy consumption in real-time, including lighting, heating, and cooling. IHG has reported energy savings of up to 15% in hotels that have implemented these systems.

%    The Le Meridien Hotel in Frankfurt, Germany, has installed smart thermostats in every guest room. This led to a 15% reduction in energy consumption and an annual cost savings of $90,000.

%    The Hilton Hotels & Resorts has implemented a “LightStay” program, which uses technology to track the hotel's energy consumption, waste, and water usage. This program has helped Hilton Hotels & Resorts to reduce their environmental impact and save costs.

%    The JW Marriott Hotel in Los Angeles, USA, has installed a building management system to optimize the energy consumption of the hotel. This has led to a 30% reduction in energy consumption and an annual cost savings of $250,000.

%    The Radisson Blu Hotel in London, United Kingdom, has implemented an energy management system to monitor and control the energy consumption of the hotel. This system has helped the hotel to reduce energy consumption by 15% and save $100,000 annually.

Come team, si è definita una soluzione unica per l'automazione degli hotel che mette al centro il cliente. L'obiettivo è quello di monitorare il consumo e di coinvolgere gli ospiti in comportamenti sostenibili attraverso l'utilizzo di tecniche di \textit{gamification} e premi. Ciò può essere realizzato senza la necessità di costi elevati e può essere installato anche in hotel datati evitando costi di ristrutturazione. Il nostro sistema premia gli ospiti per comportamenti sostenibili, per esempio tramite un sistema di punti o una classifica e potrebbe incorporare elementi di gioco, come sfide o \textit{quest}. Alcuni esempi di premi per gli ospiti potrebbero essere:
\begin{itemize}
    \item upgrade gratuito della camera;
    \item colazione gratuita;
    \item sconto sui futuri soggiorni;
    \item regalo \textit{eco-friendly} come una borraccia o una borsa da viaggio (\textit{tote bag}).
\end{itemize}
Ovviamente questi premi devono essere adattati al tipo/budget dell'hotel ed ai gusti del cliente. Questo approccio è vantaggioso sia per gli hotel, poiché riduce il consumo di energia e aumenta la soddisfazione/fedeltà degli ospiti, sia per il cliente stesso dato che acquisisce consapevolezza sull'impatto delle proprie azioni in termini di eco-sostenibilità.
%
Molti hotel hanno già programmi di fedeltà o premi a punti, tuttavia l'integrazione della tecnica di \textit{reward} per comportamenti virtuosi (sostenibili) è un concetto relativamente nuovo. Infatti alcuni hotel e associazioni si stanno muovendo verso questa direzione, per esempio:
\begin{itemize}
    \item La catena Kimpton Hotels ha avviato un programma chiamato \href{https://www.ihg.com/kimptonhotels/content/us/en/about-us/kimpton-cares/environment
    }{"EarthCare"}  che si concentra sulla riduzione dell'impatto ambientale attraverso la conservazione dell'energia/acqua, il riciclaggio/compostaggio e la promozione dell'approvvigionamento sostenibile; Offrono inoltre ai loro ospiti l'opzione di non usufruire dei servizi giornalieri di pulizia così da ricevere un credito di 5\$ al giorno da utilizzare al ristorante o al bar dell'hotel.
    \item Il programma \href{http://www.greenkeyglobal.com/home/green-key-eco-rating-2/
    }{Green Key Eco-Rating} premia gli hotel che raggiungono un certo livello di sostenibilità ambientale. Gli hotel che partecipano al programma vengono riconosciuti come "amici dell'ambiente" tramite un etichetta posta sia nel sito di Eco-Rating che in quello dell'hotel.
    \item L'InterContinental Hotels Group ha un programma chiamato \href{https://www.ihg.com/content/us/en/about/green-engage}{"IHG Green Engage"} che premia gli ospiti attraverso punti in relazione alla scelta di risparmiare energia e risorse. Questo programma consente inoltre all'hotel di monitorare le proprie prestazioni ambientali e di fissare obiettivi di miglioramento.
\end{itemize}
Queste però sono iniziative che non sfruttano le potenzialità derivanti dall'adozione di un sistema automatizzato, in genere si basano su scelte consapevoli degli ospiti. La novità della soluzione proposta è proprio quella di trovare un punto di intersezione tra i programmi/iniziative sostenibili e la tecnologia di cui il meccanismo abilitante è l'IoT combinato con la \textit{gamification}. Rispetto ai sistemi disponibili sul mercato, Ecotrip può essere installato su un qualsiasi hotel, anche se datato, a un prezzo calmierato poiché si evitano i costi di ristrutturazione.

\newpage